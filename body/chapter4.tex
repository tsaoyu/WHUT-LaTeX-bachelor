\section{已知问题和未来发展}
由于本模板没有采用class类文件来进行格式的控制,所以在格式的实现上还是使用了许多并不规范的手法。这些手法可以解决大部分论文写作中的问题,但是也造成了一点点困扰在此说明。
\subsection{已知问题}
字体控制在表格中会出现报错:这是由于模板直接使用了ctex的article类文件,而没有对于字体控制在导言部分进行再定义造成的。其解决方案可以有重写类文件,或者采用底层的CJK方案代替等。但这两个方案在实现上都不如本模板中的效果好,希望有想法的用户可以提出宝贵意见。

Bib\TeX 在\TeX Studio中工作不稳定:在引用文献时,我们需要进行至少四次编译其顺序是:\LaTeX 编译一次, bib\TeX 编译一次, 再用\LaTeX 编译两次。分别是为了生成bib文件,aux文件,文献排序,生成文献列表。可能需要用户在最后用\TeX Live 中自带的\TeX works 来手动走一遍上面的步骤。
中英摘要不能实现跳转:在本模板中采用了引用的动态链接,在目录,脚注,引用处点击可以直接跳转到对应位置。但是由于双语目录采用了非规范的语法表达,所以无法实现从目录完成跳转。
\subsection{未来发展}
武汉理工大学本科生论文的未来发展还是需要各位用户的参与,如果每一个用户都能贡献出一点关于\LaTeX 模板的想法和意见,我相信几年之后武汉理工大学本科生论文模板会成为其他高校学习和借鉴的例子。同学当自强,让我们一起来丰富完善这个模板,如果你有很好的建议或者意见请发送到 thesis@tsaoyu.com
\subsection{官方认证}
到目前为止(\today )没有武汉理工大学任何官方组织对于本模板的格式或者内容进行认证,这代表采用本模板进行的论文写作可能不被官方的论文系统接受。如在进行原创性(防抄袭)检测的时候,可能需要提供提供doc版本的论文。希望用户了解到这个潜在的风险,做好文件转换和备份的准备。本人不对任何由于使用本模板而导致的毕业论文纠纷承担任何责任!