\section{基本编译环境}
\subsection{编辑环境}

本模板采用\TeX Studio 2.6.6 作为编辑器,\TeX Live 2013套件作为编译器。
他们都是跨平台的编译软件,更重要的是他们都是完全免费的软件。 安装的顺序是先安装
\TeX Live 2013套件,再安装\TeX Studio 编辑器。 \TeX Studio编辑器的优点
有:
\begin{enumerate}
\item 清晰的组织结构,你可以在屏幕左侧看到他们
\item 便捷的自动补充功能,只要输入命令的一部分就能够完成撰写
\item 合理的宏包查看方式,右键菜单中可以找到宏包的文档
\item 贴心的实用工具,矩阵插入助手,表格编辑助手等
\end{enumerate}

当然还有其他的编译方式,比如WinEdt,LyX,Vim,甚至在线的编辑器。我也会把这份样板
上传到ShareLaTeX上,这样你就可以方便的使用在线编辑系统了。但不论你使用哪一种编辑软件或者编译器,请务必拥有他们的说明手册。这里,\LaTeX 是使用
一个个各自独立的宏作为功能的概念的,所以这些说明手册就是宏包的文档。
比如本模板使用了美国数学协会的\AmS math宏包来完成数学公式的编写,在你需要
编写数学公式你就需要找到\AmS math宏包的说明文档,文档中会告诉你这个宏包如何
使用控制语句编写数学公式。这些文档可以在网络上,或者下载的套件中找到。当然,最便捷的途径已经在上面的features里面告诉你了。
\subsection{如何使用本模板}
当你获得了所有必需的软件之后,就可以试着编译这个文档了。编译的要求是把thesis.tex设定为主文档,默认编辑器(Default compiler)则使用Xe\LaTeX 。在\TeX Studio 中的设置方法是点击Option之后选择Setup,把Build选项卡里的Default compiler 改为Xe\LaTeX 即可。如果你在编译过程中遇到红色字体提示的错误,
请拷贝错误的内容到Google中寻找解决他们的方法。此时需要你有着一定的英文阅读
能力,因为大部分\LaTeX 的文档还仍是英文的。

如果你还需要有关\LaTeX 的快速入门书籍,首先推荐阅读宏包的帮助文档,其次可以选择一些短小(200页以内)的入门小册子,中文书籍首推北京大学刘海洋编写的《\LaTeX 入门》。相信这些文档和资料可以帮助你快速的入门,发现自己所欠缺的内容之后逐渐学习最终成为\LaTeX 排版的专家。 