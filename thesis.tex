%%=======================================
%  WHUT Bachelor Thesis v 0.1b
%  
%  Tsao Yu
%  tsaoyu@tsaoyu.com
%  Last Modify Date Feb.6,2014
%%=======================================
\documentclass[c4size,a4paper]{ctexart}
                                                            
%==================== 数学符号公式 ============
\usepackage{amsmath}                 % AMS LaTeX宏包
%\usepackage{amssymb}                % 用来排版漂亮的数学公式
%\usepackage{amsbsy}
\usepackage{amsthm}
\usepackage{amsfonts}
\usepackage{mathrsfs}                % 英文花体字体
\usepackage{bm}                      % 数学公式中的黑斜体
\usepackage{bbding,manfnt}           % 一些图标,如 \dbend
\usepackage{lettrine}                % 首字下沉,命令\lettrine
\def\attention{\lettrine[lines=2,lraise=0,nindent=0em]{\large\textdbend\hspace{1mm}}{}}
\usepackage{geometry}                % 页边距调整
\geometry{top=3.5cm,bottom=3.5cm,left=3.2cm,right=3.2cm}
%\usepackage{relsize}                % 调整公式字体大小:\mathsmaller,\mathlarger
%\usepackage{caption2}               % 浮动图形和表格标题样式
%================= 基本格式预置 ===========================
\CTEXsetup[nameformat={\bfseries\zihao{-2}},
            titleformat={\bfseries\zihao{-2}}]{section}
\CTEXsetup[nameformat={\bfseries\zihao{-3}},
            titleformat={\bfseries\zihao{-3}}]{subsection}
\CTEXsetup[nameformat={\bfseries\zihao{-4}},
            titleformat={\bfseries\zihao{-4}}]{subsubsection}
\CTEXsetup[format={\Large\bfseries}]{section}
%================== 图形支持宏包 =========================
\usepackage{graphicx}                % 嵌入png图像
\usepackage{color,xcolor}            % 支持彩色文本、底色、文本框等
%\usepackage{subfigure}
%\usepackage{epsfig}                 % 支持eps图像
%\usepackage{picinpar}               % 图表和文字混排宏包
%\usepackage[verbose]{wrapfig}       % 图表和文字混排宏包
%\usepackage{eso-pic}                % 向文档的部分页加n副图形, 可实现水印效果
%\usepackage{eepic}                  % 扩展的绘图支持
%\usepackage{curves}                 % 绘制复杂曲线
%\usepackage{texdraw}                % 增强的绘图工具
%\usepackage{treedoc}                % 树形图绘制
%\usepackage{pictex}                 % 可以画任意的图形
\usepackage{hyperref}                % 交叉引用
%==================== 粘贴源代码 =====================
\usepackage{listings}                 % 粘贴源代码
%\usepackage{simpsons}       辛普森一家支持包 \bart 巴特 
%===================   正文开始    ===================
\begin{document}
\bibliographystyle{gbt7714-2005}    %论文引用格式
%===================  定理类环境定义 ===================
\newtheorem{example}{例}             % 整体编号
\newtheorem{algorithm}{算法}
\newtheorem{theorem}{定理}[section]  % 按 section 编号
\newtheorem{definition}{定义}
\newtheorem{axiom}{公理}
\newtheorem{property}{性质}
\newtheorem{proposition}{命题}
\newtheorem{lemma}{引理}
\newtheorem{corollary}{推论}
\newtheorem{remark}{注解}
\newtheorem{condition}{条件}
\newtheorem{conclusion}{结论}
\newtheorem{assumption}{假设}
%==================重定义 ===================
\renewcommand{\contentsname}{目录}     
\renewcommand{\abstractname}{\zihao{-2}摘要} % 劳动人民的智慧
\renewcommand{\refname}{参考文献}     
\renewcommand{\indexname}{索引}
\renewcommand{\figurename}{图}
\renewcommand{\tablename}{表}
\renewcommand{\appendixname}{附录}
\renewcommand{\proofname}{证明}
\renewcommand{\algorithm}{算法} 

%============== 封皮和前言 =================
%===============  封面  =================
\smallskip
\begin{center}

\vspace*{2.2cm}
\zhongsong{\zihao{1} 武汉理工大学毕业设计(论文)} \\
\vspace*{3.3cm}
\heiti{\zihao{2} 武汉理工本科论文\LaTeX 模板 }\\
\vspace*{5.5cm}

\zhongsong
\begin{tabular}{cc}
 \zihao{-2} 学院(系):&\underline{\makebox[7cm][c]{\zihao{-2}交通学院}} \\ 
 \\
 \zihao{-2}专业班级: & \underline{\makebox[7cm][c]{\zihao{-2}船舶与海洋工程1006班}} \\ 
 \\
 \zihao{-2}学生姓名: & \underline{\makebox[7cm][c]{\zihao{-2}曹宇}} \\ 
 \\
 \zihao{-2}指导教师: & \underline{\makebox[7cm][c]{\zihao{-2}徐海祥}} \\ 
 \\
\end{tabular} 
\end{center}
\thispagestyle{empty}
\clearpage
%=====================原创性声明===========
\begin{center}
\zihao{-2} \textbf{学位论文原创性声明}
\end{center}

本人郑重声明:所呈交的论文是本人在导师的指导下独立进行研究所取得的研究成果。除了文中特别加以标注引用的内容外,本论文不包括任何其他个人或集体已经发表或撰写的成果作品。本人完全意识到本声明的法律后果由本人承担。 
\begin{flushright}
\zihao{4} 作者签名:\qquad ~~~\\

年\qquad 月\qquad 日
\end{flushright}
\vskip 2cm
\begin{center}
\zihao{-2} \textbf{学位论文版权使用授权书}
\end{center}

本学位论文作者完全了解学校有关保障、使用学位论文的规定,同意学校保留并向有关学位论文管理部门或机构送交论文的复印件和电子版,允许论文被查阅和借阅。本人授权省级优秀学士论文评选机构将本学位论文的全部或部分内容编入有关数据进行检索,可以采用影印、缩印或扫描等复制手段保存和汇编本学位论文。\smallskip

本学位论文属于
\begin{tabular}[t]{l}
1、保密$ \Box$,在~~~年解密后适用本授权书  \\ 
2、不保密$ \Box$  \\ 
\end{tabular} \\
\begin{center}
(请在以上相应方框内打“$\surd”$)
\end{center}
\begin{flushright}
\zihao{4} 作者签名:  \quad\quad\quad\quad 年 \quad  月  \quad  日\\
导师签名:   \quad\quad\quad\quad 年 \quad  月 \quad   日\\
\end{flushright}
\thispagestyle{empty}
\clearpage

%%=============设计(论文)任务书===========
%\begin{center}
%\zihao{-2}\textbf{\songti 本科生毕业设计(论文)任务书} 
%\end{center}
%\smallskip
%\renewcommand{\arraystretch}{1.3}
%\begin{tabular}{lll}
%\zihao{4} \textbf{\songti 学生姓名: 曹宇} & & \zihao{4} \textbf{\songti 专业班级:\quad\quad 船海1006班} \\ 
%\zihao{4} \textbf{\songti 指导教师:徐海祥}&\makebox [3cm] & \zihao{4} \textbf{\songti 工作单位:\quad 武汉理工大学} \\ 
%\end{tabular}\\
%\begin{tabular}{lll}
%\zihao{4} \textbf{\songti 设计(论文)题目:}& \zihao{4} \textbf{\songti  武汉理工本科论文\LaTeX 模板 } &\\ 
%\zihao{4} \textbf{\songti 设计(论文)主要内容:} \\
%\end{tabular} \\ 
%\begin{enumerate}
%\item \LaTeX 环境的配置
%\item 主要字体的控制和数学公式的选用
%\item 图表和代码的粘贴
%\end{enumerate}
%\begin{tabular}{ll}
%\zihao{4} \textbf{\songti 要求完成的主要任务:}
%\end{tabular} \\ 
%\begin{enumerate}
%\item 选择合适的\TeX 编辑系统
%\item 学习如何使用控制代码完成排版
%\item 合理的安排学习和科研的时间来发展自己兴趣爱好
%\end{enumerate}
%\begin{tabular}{ll}
%\zihao{4} \textbf{\songti 必读参考资料:}
%\end{tabular}
%\begin{enumerate}
%\item \LaTeX  \quad User Manual
%\item  字体设计的艺术
%\end{enumerate}
%\begin{tabular}{lll}
%\zihao{4} \textbf{\songti 指导教师签名: }&\makebox [4cm]& \zihao{4} \textbf{\songti 系主任签名:} \\
%& & \zihao{4} \textbf{\songti 院长签名(章)}
%\end{tabular}
%\thispagestyle{empty}
%\clearpage
%%==========本科生毕业设计(论文)开题报告  =============
%\begin{center}
%\zihao{-2} \textbf{\songti 武汉理工大学}\\
%\zihao{-2} \textbf{\songti 本科生毕业设计(论文)开题报告} 
%\end{center}
%\begin{tabular}{|l|}
%\hline \rule[-2ex]{0pt}{5.5ex} \makebox[13.5cm][l]{\zihao{4} \heiti 1、目的及意义(含国内外的研究现状分析) } \\ 
%\quad \LaTeX 是国际通行的科技论文排版软件,国际上科研机构和大学都采用它写作\\
%\quad 国内著名高校都有自己的本科生\LaTeX 模板供毕业生使用\\
%\quad 但是武汉理工大学还没有本科生\LaTeX 模板可以参考\\
%\quad 人类的价值在于创造而不是索取 \\
%\hline \rule[-2ex]{0pt}{5.5ex}  \zihao{4} \heiti
%2、基本内容和技术方案\\ 
%\quad 采用GITHUB托管降低代码维护成本\\
%\quad 加入在线\TeX 编辑器的使用简介 \\
%\quad 授人以渔,注重方法和理念的引导\\
%\hline \rule[-2ex]{0pt}{5.5ex}  \zihao{4} \heiti
%3、进度安排 \\ 
%\quad 离 deadline 两个月吃喝玩乐 \\
%\quad 离 deadline 一个月吃喝玩乐 \\
%\quad 离 deadline 半个月吃喝玩乐 \\
%\quad 离 deadline 一个星期狂写论文 \\
%\hline \rule[-2ex]{0pt}{5.5ex} \zihao{4} \heiti
%4、指导教师意见 \\ 
%\quad 曹宇同学是个好同志\\
%\quad 曹宇同志是个好同学\\
%\quad 本表格是支持跨页的长表格,你可以复制上面的内容进行测试\\
%\quad 具体方法是将tabular改为 longtable然后再编译\\
%\makebox[10cm][r]指导教师签名:\\
%\makebox[12cm][r]\quad 年\quad 月\quad 日\\
%\hline 
%\end{tabular} 
%\thispagestyle{empty}

\tableofcontents 
\thispagestyle{empty}
\begin{abstract}
\vskip0.5cm
 \LaTeX 是一种高效的科技论文排版软件。普通的商业办公软件Word或者WPS有着很大
 的不同,
学习\LaTeX 科技排版不是一蹴而就的事情,需要各位多学多练方能成功。
从自身做起,脚踏实地一步一步学习基本的控制命令能够让每个用户在今后的科研道路上收获很多。相信使用\LaTeX 是一个愉悦的过程,你将会体验到编程语言的简洁和美丽。\\

\textbf{关键词:}  \LaTeX,\TeX ,毕业论文,武汉理工
\addcontentsline{toc}{section}{摘要}
\thispagestyle{empty}
\clearpage
\fontspec {Times New Roman}   %用了Times New Roman字体来美化观感
\begin{center}
\zihao {2}  \textbf{Abstract}
\end{center}
\LaTeX is a kind of high efficiency type setting software package. Far from using the common commercial software Word or WPS, you may takes a long time to practice how to type setting with\LaTeX.Learning fundamental control grammar step by step and you will earn a lot in the process of scientific research. I hope the study of \LaTeX would be a pleasure experience that in a mixture of laconic and pulchritude.\\
\textbf{Key Words:} \LaTeX,\TeX ,Bachelor Thesis, Wuhan University of Technology 
\addcontentsline{toc}{section}{Abstract}
\thispagestyle{empty}
\end{abstract}

%============== 论文正文   =================
\pagenumbering{arabic}
\section{\LaTeX 入门简介}
\LaTeX 是国际通行的格式化排版系统,在数学界和计算机科学界有着极为广泛的运用。学习\LaTeX 排版规则是每一个科研人员熟悉科研论文格式化写作,提高论文质量的不二之选。
\subsection{编辑环境}
编译环境由编辑器和编译器两个部分组成, 编辑器的功能和我们常见的写字板差不多,它能够了方便我们处理\TeX 源码明确彼此之间的篇章关系,从而提高排版效率。而编译器则是将\TeX 语言转化为计算机能够理解的二进制代码并最终呈现为我们能够阅读的PDF文档,他们之间相互分工共同完成排版任务。
\subsubsection{编辑器}
编辑器的种类非常多,有“所见即所得”的\textbf{LyX},也有Linux向的\textbf{Emacs}和\textbf{Vim},还有伪geek向的\href{http://www.sublimetext.com/}{\textbf{Sublime Text}},而我自己则偏爱IDE向的
\href{http://texstudio.sourceforge.net/}{\textbf{\TeX Studio}}.它有着一些令我爱不释手的特性,如:
\begin{enumerate}
\item 清晰的组织结构,你可以在屏幕左侧看到他们
\item 便捷的自动补充功能,只要输入命令的一部分就能够完成撰写
\item 合理的宏包查看方式,右键菜单中可以找到宏包的文档
\item 贴心的实用工具,矩阵插入助手,表格编辑助手等
\end{enumerate}

每个人都可以选择自己顺手的编辑器,如果你真的非常懒不愿意在如此多的选择中做出一个抉择那么编译器中自带的\textbf{\TeX Works}也是一个不错的选择。(其实VS Code也很不错,十分轻量,配合插件可以实现许多功能,建议使用。)
\subsubsection{编译器}
编译器一般存在于封装了宏包的各种\TeX 发行包中,按照宏包数量的多少从几十兆字节到若干个G都有。按照操作系统平台的不同,比较流行的发行包有\TeX Live,pro\TeX t和Mac\TeX . 在Windows 平台或者 Linux 平台上常用的是\href{https://www.tug.org/texlive/}{\TeX Live},如果您需要从网络上下载请选择\href{https://www.tug.org/texlive/acquire-iso.html}{ISO镜像}进行下载。国内知名大学均有镜像FTP下载站,通过他们你可以获得这个3GB左右的ISO包,安装它可以免去您下载各类宏包和寻找文档的麻烦。
\subsection{尝试编译}
\subsubsection{Windows}
安装并设置完毕软件环境之后,就可以尝试对于本论文进行编译工作。打开文件夹中的\verb|thesis.tex| 文件,将默认编译器设置为Xe\LaTeX(\TeX Studio 中依次点击Options - Configure TeXstudio - Built - Default Complier 内选择Xe\LaTeX ,\TeX works 则可以选择左上角的下拉菜单在其中找到Xe\LaTeX ),点击编译按钮就可以开始编译过程了。

正常编译结束之后,文件夹中会出现一个\verb|thesis.pdf|的文件同时编辑器也会自动打开该文件生成一个精美的预览。你可以对比自己编译出来的成果与本文件之间的差异,来确定编译器和编辑器是否已经设置妥当。
\subsubsection{Mac OS X}

在OS X系统下,由于系统内字体的区别,本模板会遇到一些编译上的问题。 我们需要手动调整一下字体的设置,以正常编译模板, 具体修改方式可以参见\href{http://www.zhihu.com/question/22906637}{知乎问答}。 

问答的第四步可能需要一些修改,
\begin{verbatim}
	cd /usr/local/texlive/2014/texmf-dist/tex/texlive/ctex
\end{verbatim}

\subsubsection{Linux}
本模板在Ubuntu 14.04 以及12.04 长期稳定支持版上均编译通过。

\subsection{简单步骤}
先\href{https://www.tug.org/texlive/acquire-iso.html}{下载}\TeX 发行包(内含编译器和相关宏包及文档),安装这个发行包大概需要20分钟左右的时间,安装期间请关闭杀毒软件以保证组件的顺利注册。
使用自带的编辑器或者下载\href{http://texstudio.sourceforge.net/}{\textbf{\TeX Studio}},作为默认编辑器使用。打开\verb|thesis.tex|,并设置编译器为Xe\LaTeX 再进行一次编译。如果遇到无法编译的问题请注意以下技术细节:

相关路径设置是否正确,在\textbf{\TeX Studio}的Options - Configure TeXstudio - Commands 中检查路径,正确的路径形式应该类似于

\begin{verbatim}
"D:/Program Files/texlive/2013/bin/win32/latex.exe"
 -src -interaction=nonstopmode %.tex
\end{verbatim}

























      %
\section{开始撰写论文}
在\LaTeX 中论文的组织形式是严格按照结构化写作的方式展开的,章节之间层次分明,段落之间关系紧密。要做到这一点就需要熟悉结构化写作的一般过程,首先需要通过\TeX 命令定义各章节的标题。
\begin{verbatim}
\section{开始撰写论文}              %对应为  第2章 开始撰写论文
\subsection{标题与正文格式控制}     %对应为   2.1 标题与正文格式的控制 
\subsubsection{字体的控制命令}      %对应为   2.1.1 字体的控制命令
\end{verbatim}
由于采用了\verb|ctex|的\verb|article|类作为论文的基本类,所以定义标题的层级最多为二级标题。当你的论文出现三级标题如\verb|2.1.1.1|的时候,请考虑修改文章层级结构以适应格式化排版的要求。(四级标题多出现于书籍以及科技专著中,毕业论文作为文档类其出现此类三级标题的情况较为罕见)。在一个低级标题之后出现的一个高级标题会使得文档当前内容跳出作用域,通过这样的方式整个文章的整体脉络就可以很清晰地显现出来。
\subsection{字体字号的控制}
字体字号的处理是借助了\href{http://mirror.hust.edu.cn/CTAN/language/chinese/ctex/ctex.pdf}{ctex}宏包实现的,仔细阅读该文档你能在中文格式处理方面节省许多时间。在宏包中对于处理字体和字号的方法进行详细的阐述。在我们熟知的排版系统中,形式和内容是一个密不可分的整体,两者相生相伴无法分离。从我们写下一段话,并选中这段话然后再设定字体和字号开始形式已经开始附加到我们想要表达的内容中了。但是在\LaTeX 中,所有的内容(也就是正文及相关附录)是不包含任何关于格式的信息的。这样就做到了形式与内容的彻底分离,是\LaTeX 区别于任何一个排版系统的根本原因。

实现内容与形式的剥离是一个痛苦的过程,我们需要摒弃我们懒惰的直觉并开始高度抽象化的思考,通过这样一个过程等到内容与形式再度统一。
\subsubsection{字体}
根据中文汉字支持宏包ctexart的参考文档,模板中预置的常用字体一共有五种,他们分别是:宋体,黑体,仿宋,楷书, 隶书。对应的控制方式如下:

\begin{table}[htbp]
\begin{center}
\begin{tabular}{cccccc}
\toprule 
 { 宋体} & { 黑体} & { 仿宋} & { 隶书} & {楷书} &{华文中宋}\\ 
\midrule 
\textbackslash songti &\textbackslash  heiti  & \textbackslash fangsong & \textbackslash lishu & \textbackslash kaishu & \textbackslash zhongsong\\ 
\bottomrule 
\end{tabular}  
\end{center}
\end{table}
使用华文中宋请自行下载安装。以上字体基本满足了武汉理工大学本科生毕业论文中所要求的字体的需求。


\subsubsection{字号}
使用\verb|\zihao{4}|命令来规定四号字体,在前面加负号表示小四\verb|\zihao{-4}|.
\subsection{图片及表格的处理}

\subsubsection{在文档中加入图片}
理论上\LaTeX 可以处理各种各样的图片类型从jpeg到bmp,从pdf到eps都是可以接受的图片处理类型。选择合理的图片类型会提高论文的整体观感,使得最终的排版效果更为优良。而其中以无损压缩格式为优先推荐,原生pdf图片,原生eps图片都是最优的选择。如果实在无法找到矢量图,可以退而求其次地采用png图片或者jpeg格式的图片。
\begin{itemize}
\item \textbf{取人玫瑰手留余香}~~~~使用他人图片时记得标注出处和明显的引用。
\item \textbf{掌握一种数据绘图软件}~~~~Python, MATLAB, Mathematica 都是不错的选择
\item \textbf{探索示意图绘制的方法}~~~~指的是流程图,二维或三维线图,推荐Ipe editor, TiKZ, 以及 Microsoft Visio Ink-scape
\end{itemize}
图片插入范例
\begin{figure}[thbp!]
\centering
\includegraphics[width=0.4\linewidth]{figure/IMG_1832}
\caption{\LaTeX 字号错误使用范例}
\label{fig:IMG_1832}
\end{figure}

为了插入这样的图片,我们使用了如下的代码:
\begin{verbatim}
\begin{figure}[thbp!]
\centering
\includegraphics[width=0.6\linewidth]{figure/IMG_1832}
\caption{\LaTeX 字号错误使用范例}
\label{fig:IMG_1832}
\end{figure}
\end{verbatim}
其中第一行的\verb|[thbp!]|是用来规定图片位置的命令\verb|t|表示顶部,\verb|h|表示这里,\verb|b|表示底部,\verb|p|则表示“随便哪儿!”\verb|!|则表示“就是这里!” 在第三行中,规定了图片的尺寸,其方式为限定尺寸宽度为0.6倍行线宽。最后是图片的标题和它的标号,有了它们可以很方便地引用一个图片。

在文档中,虽然我们规定了图片所安放的位置和相对应的顺序,但是图片最终在文档中所呈现的位置和代码中的还是会有差距。这是由于所有的图片实际上都是“浮动” 环境,在设置了图片的大小之后实际上最终的位置还是由文字结束之后可以容纳图片的空间位置所决定的。 如果文档末尾空间不足以填充图片,那么排版系统会自动先将文字填充于这个部分然后再放置我们想要的图片。图片的位置时常让我们感到困惑,如果遇到图片位置的问题可以有几个思路参考:
\begin{itemize}
	\item 更改图片大小,或者宽度。由于大多数情况下我们需要图片等比例缩放,实际上修改宽度和修改图片大小是一样的原理。
	\item 新增一个新的页面,容纳过多的图片。
	\item 合理安排图片的数量,避免做“插图大师”。科研文章都是为了内容服务的,切莫为了字数要求,页数要求而恶意灌图。
\end{itemize}

如图(\ref{fig:IMG_1832})中显示了一个错误的字号显示的方法。
\begin{figure}[h]
\centering
\tdplotsetmaincoords{70}{110}
\begin{tikzpicture}[auto,scale=1,tdplot_main_coords]
\draw[thick,->] (0,0,0) -- (3,0,0) node[anchor=north east]{$x$};
\draw[thick,->] (0,0,0) -- (0,3,0) node[anchor=north west]{$y$};
\draw[thick,->] (0,0,0) -- (0,0,3) node[anchor=south]{$z$};
\draw[thick,->,color=red] (1,0,0)--(1,1.5,0.8) node[midway,name=a]{$ \vec{a} $};
\draw[thick,->,color=red] (1,0,0)--(1,-0.8,1.5) node[midway,name=b]{$ \vec{b} $};
\end{tikzpicture}
\caption{三维向量旋转示意}
\label{fig:3Drot}
\end{figure}

而图(\ref{fig:3Drot})则是用TikZ语言做的图片,比较清晰明了。
\subsubsection{在文档中加入表格}
三线表的使用,见如下代码
\begin{verbatim}
\begin{table}[thbp]
\caption{状态估计算法比较}
\begin{center}
\begin{tabular}{cccc}
\hline  & 卡尔曼滤波 & 神经网络滤波 & 被动无源滤波 \\ 
\hline 模型类型 & 线性 & 线性 & 非线性  \\ 
 参数调校 & 大量 & 几乎没有 & 合理 \\ 
 稳定性 & 满足全局稳定性 & 依赖于模型 & 满足子系统稳定性 \\ 
 算法开销 & 低且可以借助硬件实现  & 高且大量依赖软件平台  & 低且可以借助硬件实现  \\ 
 \hline
\end{tabular} 
\end{center}
\label{tb:filter}
\end{table}
\end{verbatim}
\verb|tabular|后的\verb|cccc|表示四个居中的行元素,\verb|llll|则表示四个居左的行元素,\verb|&|分割行元素,\verb|\\|分割列元素,一个\verb|\hline|就是一条线。
\begin{table}[thbp]
\caption{状态估计算法比较}
\begin{center}
\begin{tabular}{cccc}
\hline  & 卡尔曼滤波 & 神经网络滤波 & 被动无源滤波 \\ 
\hline 模型类型 & 线性 & 线性 & 非线性  \\ 
 参数调校 & 大量 & 几乎没有 & 合理 \\ 
 稳定性 & 满足全局稳定性 & 依赖于模型 & 满足子系统稳定性 \\ 
 算法开销 & 低且可以借助硬件实现  & 高且大量依赖软件平台  & 低且可以借助硬件实现  \\ 
 \hline
\end{tabular} 
\end{center}
\label{tb:filter}
\end{table}

如果遇到表格比较复杂的情况,也不必抓耳挠腮,可以使用诸如\href{http://www.tablesgenerator.com}{在线表格编辑器}之类的小工具帮助我们完成工作。
\subsection{数学公式}

美观简洁的数学公式是\LaTeX 中的一大特点,按照数学公式的类型可以分为标号公式和不标号公式两者。不标号公式有有行内公式和行间公式的两种类型分别类似于,行内公式$ e^{i\pi}+1=0 $ 和\[ \dfrac{d\vec{G}}{dt}=\dot{G_x}\vec{i}+\dot{G_y}\vec{j}+\dot{G_z}\vec{k}+G_x\dot{\vec{i}}+G_y\dot{\vec{j}}+G_z\dot{\vec{k}} \],
分别使用美元符号和方括号命令来表示。 通常在学术论文中正文里的重要公式需要编号,编号的公式类型主要有\verb|equation|,\verb|align|,\verb|split|,\verb|eqnarray| 等类型,能够实现等式,方程组,跨行公式的显示。具体的使用方式见\ref{example}
\subsubsection{一个简单的例子}\label{example}
船舶运动中所涉及的力和速度都可以理解为矢量,按照矢量旋转的方法可以对于坐标系统进行转化。
\begin{lemma}
\label{2Drot} 
存在一个旋转矩阵使得任何两个模相同的二维向量相互转换
\end{lemma}
\begin{proof}
设定向量$\vec{X}=(a_1,b_1),\vec{Y}=(a_2,b_2)$ 存在$ J $使得$ XJ=Y $同时$X=YJ^{-1}$,其中\[  \sqrt{a_1^{2}+b_1^{2}}=\sqrt{a_2^{2}+b_2^{2}}=R \] 
由线性方程组的解可知,当$rank(A,Y)=rank(A)=2$时线性方程组有唯一解,此时矩阵$ J $定义为\textit{旋转矩阵},同时$ J^{-1} $定义为\textit{逆旋转矩阵}.
\end{proof}
\begin{theorem}
\label{rotM}
平面旋转矩阵$ J $只和两向量之间的夹角$ \theta $有关
\end{theorem}

\begin{proof}
$ a_1=Rsin\alpha , b_1=Rcos\alpha~~ .~~ a_2=Rsin\beta ,b_2=Rcos\beta $
展开  $ a_2 $可以得到
\[a_2=Rsin\beta=Rsin(\alpha+\theta)=R(sin\alpha cos\theta +cos\alpha sin \theta)\]  
将$cos\alpha=\dfrac{a_1}{R},sin\alpha=\dfrac{b_1}{R}$代入可以得到\[a_2=a_1 cos\theta-b_1 sin\theta \] 同理\[ b_2=a_1 sin\theta+b_1 cos\theta \] 转换为矩阵形式则为
 \begin{align} \begin{bmatrix}
 a_2\\b_2 
\end{bmatrix}=
\begin{bmatrix}
 cos\theta&-sin\theta\\sin\theta& cos\theta
\end{bmatrix}  
\begin{bmatrix}
 a_1\\b_1 
\end{bmatrix}\end{align}
 最终可以得到
 \begin{align}
 \label{Jc}
  {J}_c=\begin{bmatrix} cos\theta&-sin\theta\\sin\theta& cos\theta
\end{bmatrix} 
\end{align}
 逆时针旋转时

\begin{align}
\label{Jcc}
{J}_{cc} =\begin{bmatrix} cos\theta&sin\theta\\-sin\theta& cos\theta
\end{bmatrix}
\end{align}
\end{proof}
\begin{theorem}
任何两个模相同的三维向量,可以通过旋转矩阵相互转化
\end{theorem}
\begin{figure}[h]
\centering
\tdplotsetmaincoords{70}{110}
\begin{tikzpicture}[auto,scale=1,tdplot_main_coords]
\draw[thick,->] (0,0,0) -- (3,0,0) node[anchor=north east]{$x$};
\draw[thick,->] (0,0,0) -- (0,3,0) node[anchor=north west]{$y$};
\draw[thick,->] (0,0,0) -- (0,0,3) node[anchor=south]{$z$};
\draw[thick,->,color=red] (1,0,0)--(1,1.5,0.8) node[midway,name=a]{$ \vec{a} $};
\draw[thick,->,color=red] (1,0,0)--(1,-0.8,1.5) node[midway,name=b]{$ \vec{b} $};
\end{tikzpicture}
\caption{三维向量旋转示意}
\label{fig:3Drot}
\end{figure}

\subsection{呈列代码}
采用listing宏包可以列代码,在控制文件导演区可以更改listing的设置
来符合MATLAB,Python,C++等不同语言的需求。
\begin{lstlisting}[language=C]
int main(int argc, char ** argv)
{
printf("Hello world!\n");
return 0;
}
\end{lstlisting}
\section{进阶功能的使用}
\subsection{参考文献和引用}
没有人希望看到自己辛苦撰写的毕业论文被判定为抄袭,而抄袭与合理的引用的差别仅仅在于作者有没有在文中明确的标注自己所引用文献的出处。而\TeX 作为严谨的科学论文排版体系,对于文献引用进行了完善的设计。用户需要学习如何通过合理的手段取得这些文献,并且恰当地引用他们来佐证自己的观点。
相信通过这样的学习,每个用户都能严肃而认真地对待论文引用的问题。
\subsubsection{来源——高效的文献管理软件}

随着网络和电子技术的发展,图书馆的信息收集和服务等也越来越多的实现了电子化和网络化。各种类型的图书馆纷纷建立自己的电子信息服务系统来追赶这一发展趋势。与此同时,由于信息传播的路径越来越多和越来越便捷,文献服务的时效性也变得非常突出,用户需要其机构图书馆能够提供及时的文献动态服务。同时,众多的学校和研究机构图书馆开始购买国内外各种电子资源数据库,面对海量的研究信息,往往让缺少文献信息收集和管理的研究者却步,这在一定程度上也影响了学校购买的昂贵的电子资源的使用效率和最终的研究成果产出。
面对海量文献信息的电子化,信息过滤和管理变得极其重要,为了提高研究者对电子资源的使用效率,为了帮助研究者有效管理和利用这些电子文献,文献管理类软件应运而生。作为中国大陆和台湾等地高校和研究院所非常受欢迎的文献管理软件NoteExpress的开发商——北京爱琴海乐之技术有限公司,我们为众多研究人员提供了专业的文献信息管理解决方案。NoteExpress已经成为中国文献管理软件市场上的第一品牌,拥有绝对领先的软件性能和市场优势。NoteExpress 围绕科学研究最核心的文献信息,为用户提供了信息导入、过滤、全文下载,以及众多的管理功能,可以大大提高研究者的文献管理和研究效率。

这是一个很好用的软件,学生版本的最低售价是198元,但是我们学校买了他的集团版本,所以只要用户拥有武汉理工大学的注册在籍学生身份就能完全免费的使用。这也是任何一个决心搞科研必备的同学素质:既然选择了科研事业,就要有向全人类要吃要喝的决心。
\subsubsection{实现——Bib\TeX 系统}
Bib\TeX 是\LaTeX 独有的文献管理系统,通过Bib\TeX 我们可以很方便的完成文献的引用。比如说现在我们要引用一篇题为《LaTeX与方正书版排版数学论文探讨》的文章,传统的Bib\TeX 做法是去网络上手动的把他的信息给扒下来,俗称“人肉论文引用机”。但这样做是我们后面工作的基础,因为巧妇难为无米之炊我们需要引用文章的基本信息这样才能够在合适的位置显示他们。
在Bib\TeX 中,我们需要的基本信息有题名,作者,期刊名,出版年份,甚至出现的页数等等信息
得到这些信息之后把他们写到Bib\TeX 文件里,例子里面的信息应该呈现出这样的样式。
\begin{lstlisting}[language=TeX]
@article{
王勇姚萍-197,
   Author = { 王勇 and 姚萍 and 王岚 and 庞立},
   Title = {LaTeX 与方正书版排版数学论文探讨},
   Journal = { 中国科技期刊研究} ,
   Year = {2012} }

\end{lstlisting}
拥有这样的信息之后就可以在文章中出现他们引用的地方用\textbackslash cite命令完成引用,并且在文章最后的引用文献中按照规范(GBT7714-2005N)列出他们。假设我们引用了《LaTeX与方正书版排版数学论文探讨》中的内容“LaTeX不知道要比方正高明到哪里去了!"\cite{王勇姚萍-197},那么以上就应该是你所见到文章中的样式。这里再举一个例子,关于字体的设计我们有很多话要说。\cite{吴昉张页-198}
\subsubsection{结合——管理自己的文献}
首先用户需要到NoteExpress网站上自行下载武汉理工大学专版的软件,安装完毕后打开软件。
\begin{figure}[htbp]
\centering
\includegraphics[width=7cm]{figure/1}
\caption{启动软件}
\label{fig:1}
\end{figure} \\

打开软件之后选择检索菜单里面的在线检索,根据自己的文档类型选择合适的数据库。
\begin{figure}[htb]
\centering
\includegraphics[width=7cm]{figure/2}
\caption{检索文档}
\label{fig:2}
\end{figure} \\

对于非校园网还需要设置一下代理来完成对于校内资源的访问。
\begin{figure}[htbp]
\centering
\includegraphics[width=7cm]{figure/3}
\end{figure}\\

设置参数,图中学号应为饭卡号,在此更正。
\begin{figure}[htbp]
\centering
\includegraphics[width=7cm]{figure/4}
\end{figure}\\

找到所需要的参考文献,并放置到合适的位置如图书,期刊等。
\begin{figure}[htbp]
\centering
\includegraphics[width=7cm]{figure/5}
\end{figure}\\

此时就可以在主程序窗体中看到所引用的文献信息了,
\begin{figure}[htbp]
\centering
\includegraphics[width=7cm]{figure/7}
\end{figure}
引用时:右键单击该条目,复制出Bib\TeX 引用的信息,粘贴到所引用位置即可。
在本模板的figure文件夹下有后续步骤的图片示意。用户可以对照NoteExpress使用手册,以及这几个示意图完成设置使用的过程。

\subsection{在线编辑和共享}
在线编辑为用户提供了脱离本地环境的编译体验,也方便了文档进行共享和传递。通常来说,用户最终呈现的毕业论文应该是经过编译之后的pdf文件。这个文件满足一般设备的需求,可以在个人电脑,平板电脑,电子书阅读器等设备上获得优良的观感体验。更重要的是可以按照文档所见的样子通过打印机获得质量上乘的纸质副本,满足纸质论文的一切需求。\\

但是在\LaTeX 的使用过程中,我们可能会面临多人撰写同一个论文的情况,比如第一作者完成前言和导论的内容,第二作者完善第一章,第三作者帮助进行图标的处理。这样就需要我们传递原始的\TeX 原档来进行合作,这样做缺点非常明显:每个人的处理步骤不同步导致文档在整体上的形式不符合要求。要解决这个问题,就需要诸如本模板之类的模板文件的帮助。用户在请求共享和协作的时候,应该给所有协作者分发模板的全部内容。同时协作者应该帮助完成body里的内容,而不允许修改thesis.tex这个控制文件。用户在收到协作者的body文件之后再用include的命令把他们写入到总体框架中去。这种方法是国际期刊通行的处理方法,用户可以借鉴使用。\\

除此之外,我们还可以利用GITHUB来帮助我们进行论文的协作写作,每个独立的作者
应该拥有自己的协作者账号。在对于的论文项目下,采用merge的形式进行文档的增补和修订,这样做的好处在于GITHUB拥有完善的修订记录,可以很方便的对于不同的修订内容进行采纳或拒绝。这也给了用户给大的自由度,他们可以对于控制文件的结构提出修改,从而使论文的样式和结构更加完美。
\section{已知问题和未来发展}
由于本模板没有采用class类文件来进行格式的控制,所以在格式的实现上还是使用了许多并不规范的手法。这些手法可以解决大部分论文写作中的问题,但是也造成了一点点困扰在此说明。
\subsection{已知问题}
字体控制在表格中会出现报错:这是由于模板直接使用了ctex的article类文件,而没有对于字体控制在导言部分进行再定义造成的。其解决方案可以有重写类文件,或者采用底层的CJK方案代替等。但这两个方案在实现上都不如本模板中的效果好,希望有想法的用户可以提出宝贵意见。

Bib\TeX 在\TeX Studio中工作不稳定:在引用文献时,我们需要进行至少四次编译其顺序是:\LaTeX 编译一次, bib\TeX 编译一次, 再用\LaTeX 编译两次。分别是为了生成bib文件,aux文件,文献排序,生成文献列表。可能需要用户在最后用\TeX Live 中自带的\TeX works 来手动走一遍上面的步骤。
中英摘要不能实现跳转:在本模板中采用了引用的动态链接,在目录,脚注,引用处点击可以直接跳转到对应位置。但是由于双语目录采用了非规范的语法表达,所以无法实现从目录完成跳转。
\subsection{未来发展}
武汉理工大学本科生论文的未来发展还是需要各位用户的参与,如果每一个用户都能贡献出一点关于\LaTeX 模板的想法和意见,我相信几年之后武汉理工大学本科生论文模板会成为其他高校学习和借鉴的例子。同学当自强,让我们一起来丰富完善这个模板,如果你有很好的建议或者意见请发送到 thesis@tsaoyu.com
\subsection{官方认证}
到目前为止(\today )没有武汉理工大学任何官方组织对于本模板的格式或者内容进行认证,这代表采用本模板进行的论文写作可能不被官方的论文系统接受。如在进行原创性(防抄袭)检测的时候,可能需要提供提供doc版本的论文。希望用户了解到这个潜在的风险,做好文件转换和备份的准备。本人不对任何由于使用本模板而导致的毕业论文纠纷承担任何责任!
%=============  致谢  ======================
\include{body/chapter5}
%============= 参考文献 =====================
\addcontentsline{toc}{section}{参考文献}
\bibliography{bibfile}

\clearpage
\end{document}
%%%%%%%%%% 结束 %%%%%%%%%%